\documentclass{beamer}
\mode<presentation>
\usepackage[utf8]{inputenc}
%\usepackage{listings}
\usepackage{hyperref}
\usepackage{verbatim}
\usepackage{graphics}
\uselanguage{french}
\usetheme{Warsaw}
\usecolortheme{orchid}

\title{RACER}
\subtitle{Rapid and Accurate Correction of Errors in Reads}
\date{23 septembre 2015}
\author{Delphine Poux, Alexandre Gerussi, Valentin Owczarek}
\institute{M2 MOCAD - Sciences du Vivant}

\begin{document}

\begin{frame}
\titlepage
\end{frame}

\begin{frame}
\frametitle{Plan de l'exposé}
\tableofcontents[pausesections]
\end{frame}

\section{Évaluation des performances de correction}
\begin{frame}
\frametitle{Évaluer les performances de correction}
\begin{itemize}
    \item plusieurs stratégies possibles
    \begin{itemize}
        \item mapper les reads sur le génome de référence
        \item utiliser les reads dans d'autres applications (alignement\dots)
    \end{itemize}
    \item méthode de RACER:
    \begin{itemize}
        \item un read est cherché de manière exacte sur le génome de référence
        \item estampillé "correct" s'il est trouvé, "incorrect" sinon
        \item calcul de la somme $e_b$ et $e_a$ des longueurs des reads {\em incorrects} avant et après correction
        \item performance = taux de variation entre $e_b$ et $e_a$.
    \end{itemize}
\end{itemize}
\end{frame}

\section{Résultats et comparaisons}
\begin{frame}
    \frametitle{Jeux de données}
    \begin{itemize}
        \item utilisation de jeux de données réelles
        \item les performances sur les données artificielles se généralisent mal
        \item artificielles = plus simples à corriger
    \end{itemize}
    \begin{centering}
        \includegraphics<1>[width=11cm]{donnees.jpg}%
    \end{centering}
\end{frame}

\begin{frame}
    \frametitle{Mise en oeuvre}
    \begin{itemize}
        \item AMD Opteron 24 coeurs et 98 GB de RAM
        \item comparaison de la vitesse, de l'espace mémoire et des performances de correction
        \item vitesse et espace mémoire sont normalisés
        \item exécution en série et en parallèle 
    \end{itemize}
\end{frame}

\begin{frame}
    \frametitle{Résultats}
    \begin{centering}
        \includegraphics<1>[width=11cm]{resultats.jpg}%
    \end{centering}
\end{frame}
\end{document}
