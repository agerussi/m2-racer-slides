\documentclass{beamer}
\mode<presentation>
\usepackage[utf8]{inputenc}
%\usepackage{listings}
\usepackage{hyperref}
\usepackage{verbatim}
\usepackage{graphics}
\uselanguage{french}
\usetheme{Warsaw}
\usecolortheme{orchid}

\title{RACER}
\subtitle{Rapid and Accurate Correction of Errors in Reads}
\date{23 septembre 2015}
\author{Delphine Poux, Alexandre Gerussi, Valentin Owczarek}
\institute{M2 MOCAD - Sciences du Vivant}

\begin{document}

\begin{frame}
\titlepage
\end{frame}

\begin{frame}
\frametitle{Plan de l'exposé}
\tableofcontents
\end{frame}

\section{Introduction}

\frame{
\frametitle{Séquençage nouvelle génération}
\begin{itemize}
\item Sanger et al. 1977
\begin{itemize}
\item première génération
\item sans erreur
\item a ses limites (lent)
\end{itemize}
\item Nouvelle génération de séquençage
\begin{itemize}
\item Plus rapide, de moins en moins cher, produit des erreurs
\item Projets ambitieux
\item Illumina/Solexa, Roche/454, Life/APG’s SOLiD, Helicos BioSciences’ HeliScope, Pacific Biosciences, Life’s Ion Torrent
\end{itemize}
\end{itemize}
}

\frame{
\frametitle{Bio-informatique \& les erreurs de séquençage}
\begin{itemize}
\item Nécessite alignement de haute qualité
\item Illumina : erreurs dans un read sur 2 $\rightarrow$ corrections
\item Substitutions principalement
\item Mécanismes de corrections internes
\item Idée principale : utiliser le fait que le read a été séquencé un certain nombre de fois
\item Différence : comment faire l'analyse
\end{itemize}
}

\frame{
\frametitle{Corrections de séquences}
\begin{itemize}
\item si nombre de k-mers $>$ seuil : on garde $\rightarrow$ SHREC,
HiTEC, HSHREC et PSAEC
\item Corrige les k-mers de distance min pour dépasser le seuil : CUDA, Quake, Reptile et Hammer
\item Alignement de séquence multiple : Coral, ECHO et MyHybrid
\item Arbres des suffixes, tableau des suffixes
\end{itemize}
}

\section{Évaluation des performances de correction}
\begin{frame}
\frametitle{Évaluer les performances de correction}
\begin{itemize}
    \item plusieurs stratégies possibles
    \begin{itemize}
        \item mapper les reads sur le génome de référence
        \item utiliser les reads dans d'autres applications (alignement\dots)
    \end{itemize}
    \item méthode de RACER:
    \begin{itemize}
        \item un read est cherché de manière exacte sur le génome de référence
        \item estampillé "correct" s'il est trouvé, "incorrect" sinon
        \item calcul de la somme $e_b$ et $e_a$ des longueurs des reads {\em incorrects} avant et après correction
        \item performance = taux de variation entre $e_b$ et $e_a$.
    \end{itemize}
\end{itemize}
\end{frame}

\section{Résultats et comparaisons}
\begin{frame}
    \frametitle{Jeux de données}
    \begin{itemize}
        \item utilisation de jeux de données réelles
        \item les performances sur les données artificielles se généralisent mal
        \item artificielles = plus simples à corriger
    \end{itemize}
    \begin{centering}
        \includegraphics<1>[width=11cm]{donnees.jpg}%
    \end{centering}
\end{frame}

\begin{frame}
    \frametitle{Mise en oeuvre}
    \begin{itemize}
        \item AMD Opteron 24 coeurs et 98 GB de RAM
        \item comparaison de la vitesse, de l'espace mémoire et des performances de correction
        \item vitesse et espace mémoire sont normalisés
        \item exécution en série et en parallèle 
    \end{itemize}
\end{frame}

\begin{frame}
    \frametitle{Résultats}
    \begin{centering}
        \includegraphics<1>[width=11cm]{resultats.jpg}%
    \end{centering}
\end{frame}
\end{document}
